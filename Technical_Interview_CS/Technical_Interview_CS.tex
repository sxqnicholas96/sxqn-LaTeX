%!TEX root = ./Technical_Interview_CS.tex

\documentclass[12pt, a4paper]{article}
\usepackage{geometry}
\usepackage{mathtools}
\usepackage{enumitem}
\usepackage{fullpage}
\usepackage{graphicx}
\usepackage{amsmath}
\usepackage{hyperref}
\usepackage{fancyvrb,xcolor}

\newcommand{\code}[1]{\texttt{#1}}

\definecolor{test}{gray}{0.93}

\newenvironment{lcverbatim}
 {\SaveVerbatim{cverb}}
 {\endSaveVerbatim
  \flushleft\fboxrule=0pt\fboxsep=.5em
  \colorbox{test}{%
    \makebox[\dimexpr\linewidth-2\fboxsep][l]{\BUseVerbatim{cverb}}%
  }
  \endflushleft
}

\hypersetup{
    colorlinks,
    citecolor=black,
    filecolor=black,
    linkcolor=black,
    urlcolor=black
}

\newcommand\Perm[2][^n]{\prescript{#1\mkern-2.5mu}{}P_{#2}}
\newcommand\Comb[2][^n]{\prescript{#1\mkern-0.5mu}{}C_{#2}}

\setlength{\parindent}{0pt} % Set paragraph indent to 0 spaces

\graphicspath{./}
\geometry{a4paper, margin=0.7in}

\begin{document}
\noindent

\tableofcontents

\pagebreak

\addcontentsline{toc}{section}{Software Engineering Principles}
\section*{Software Engineering Principles}

\pagebreak
\addcontentsline{toc}{section}{Concepts in Operating Systems}
\section*{Concepts in Operating Systems}

\pagebreak
\addcontentsline{toc}{section}{Languages}
\section*{Languages}
\addcontentsline{toc}{subsection}{JavaScript}
\subsection*{JavaScript}
\subsubsection*{What is the \code{prototype} object in JavaScript}
JavaScript is often described as a prototype-based language.
Objects can have a \code{prototype} object, which acts as a template object that it inherits methods and properties from.
In JavaScript, a link is made between the object instance and its prototype (\code{\_\_proto\_\_} property, derived from \code{prototype} property from the constructor)

\subsubsection*{What are the primitive data types in JavaScript}
There are 7 primitive data types: \code{string}, \code{number}, \code{bigint}, \code{undefined}, \code{boolean}, \code{symbol} and \code{null}.

\subsubsection*{What are the keywords \code{async}/\code{await} and the \code{Promise} object used for}
The \code{async} keyword is always used to define an asynchronous function, which would then always return a \code{Promise} object.\newline
The \code{await} keyword can \textbf{only be used in defined, asynchronous functions}, makes JavaScript wait until that promise resolves, and then returns the result.
\code{await} \textbf{only} suspends function execution in which it has been used.
In the meantime, the JavaScript can handle other jobs like executing scripts or handle events etc.\newline
The \code{Promise} object represents the eventual completion and failure of an asynchronous operation and its resulting value.
A \code{Promise} can be in one of 3 states:
\begin{itemize}
    \item \code{pending} - the asynchronous action is pending resolution, neither fulfilled nor rejected
    \item \code{fulfilled}/\code{resolved} - the asynchronous action has completed successfully
    \item \code{rejected} - The operation has failed
\end{itemize}

\subsubsection*{Why are callbacks used extensively in JavaScript}
Callbacks are used in JavaScript to execute code after an asynchronous action has resolved.

\subsubsection*{What is Babel and what is its use in JavaScript projects}
Babel is a JavaScript compiler toolchain that helps developers transpile JavaScript code into backwards-compatible versions of itself to accommodate older browsers and environments.
This is because older browser versions often are unable to support newer versions of JavaScript.

\subsubsection*{What is \code{Promise.all} and \code{Promise.race}}
\code{Promise.all} accepts an array of promises, and attempts to fulfill all of them.
Exits early if just 1 promise gets rejected.\newline
\code{Promise.race} accepts an array of promises, but only returns the \code{Promise} that returns first.

\subsubsection*{What is currying in JavaScript?}
Currying is a technique of evaluating a function with multiple arguments, into a sequence of functional calls, each with a single arguments.\newline
Currying works by natural closure, where the nested functions retain access to variables in higher order functions.
Hence the innermost function have access to all variables.

\subsubsection*{What is the concept of closures in JavaScript}
The concept of closure refers to the combination of a function bundled together with references to its surrounding state or lexical environment, within which the function was declared.
Whenever you create a function within another function, you have created a closure.
In JavaScript, closures are created every time is created, at function creation time.
\begin{lcverbatim}
    function makeFunc() {
            var name = 'Mozilla';
            function displayName() {
                    alert(name);
                }
            return displayName
        }
    var myFunc = makeFunc();
    myFunc();
\end{lcverbatim}

As in the above code, the instance of displayName maintains a reference to its lexical environment, and hence is able to access the variable \code{name} in the outer function.

% TODO
\subsubsection*{Compare Global scope vs Local scope vs Function scoping vs Block-scoping}
\textbf{Global scope} refers to variables declared outside all functions or curly braces.\newline
\textbf{Local scope} refers to variables declared that is only accessible in a specific part of the code.
\begin{itemize}
    \item \textbf{Function scope} refers to lexical environment enclosed within a function.
    \item \textbf{Block scope} refers to the lexical environment enclosed within a curly brace.
\end{itemize}

\subsubsection*{What does hoisting mean in JavaScript}
Hoisting is a JavaScript mechanism where variables and function declarations are moved to the top of their scope before code execution.

\subsubsection*{What is the \code{new} keyword in JavaScript for?}

\subsubsection*{Difference between Class Expressions and Class Declarations}
\begin{lcverbatim}
class Rectangle {
    constructor(height, width) {
        this.height = height;
        this.width = width;
    }
}
\end{lcverbatim}


\subsubsection*{Difference between \code{Object} and \code{Map}}

\subsubsection*{Difference between \code{const}, \code{let} and \code{var}}

\subsubsection*{What's the difference between function declarations, function expressions and arrow functions? In which contexts would you use each type?}
https://mariusschulz.com/blog/function-definitions-in-javascript


\pagebreak

\addcontentsline{toc}{subsection}{C/C++}
\subsection*{C/C++}


\pagebreak
\addcontentsline{toc}{section}{Frameworks}
\section*{Frameworks}
\subsection*{React}

\end{document}
%!TEX root = ./question_bank.tex

\documentclass[12pt, a4paper]{article}
\usepackage{geometry}
\usepackage{mathtools}
\usepackage{enumitem}
\usepackage{fullpage}
\usepackage{graphicx}
\usepackage{amsmath}
\usepackage{hyperref}
\usepackage{fancyvrb,xcolor}

\newcommand{\code}[1]{\texttt{#1}}

\definecolor{test}{gray}{0.93}

\newenvironment{lcverbatim}
 {\SaveVerbatim{cverb}}
 {\endSaveVerbatim
  \flushleft\fboxrule=0pt\fboxsep=.5em
  \colorbox{test}{%
    \makebox[\dimexpr\linewidth-2\fboxsep][l]{\BUseVerbatim{cverb}}%
  }
  \endflushleft
}

\hypersetup{
    colorlinks,
    citecolor=black,
    filecolor=black,
    linkcolor=black,
    urlcolor=black
}

\newcommand\Perm[2][^n]{\prescript{#1\mkern-2.5mu}{}P_{#2}}
\newcommand\Comb[2][^n]{\prescript{#1\mkern-0.5mu}{}C_{#2}}

\setlength{\parindent}{0pt} % Set paragraph indent to 0 spaces

\graphicspath{./}
\geometry{a4paper, margin=0.7in}

\begin{document}
\noindent

\tableofcontents

\pagebreak

\addcontentsline{toc}{subsection}{JavaScript}
\subsection*{JavaScript}
\subsubsection*{What is the \code{prototype} object in JavaScript}
JavaScript is often described as a prototype-based language.
Objects can have a \code{prototype} object, which acts as a template object that it inherits methods and properties from.
In JavaScript, a link is made between the object instance and its prototype (\code{\_\_proto\_\_} property, derived from \code{prototype} property from the constructor)

\subsubsection*{What are the primitive types in JavaScript}

\subsubsection*{What is async, await and Promises}

\subsubsection*{What is the difference between asynchronous and synchronous function in JavaScript}

\subsubsection*{What is Babel? Why is it needed?}

\subsubsection*{What is Promise.all and Promise.race}

\subsubsection*{What is hoisting in JavaScript? What are its implications? }

\url{https://www.w3schools.com/js/js_hoisting.asp}

\url{https://www.freecodecamp.org/news/function-hoisting-hoisting-interview-questions-b6f91dbc2be8/}

\subsubsection*{What's the difference between function declarations, function expressions and arrow functions? In which contexts would you use each type? }

\url{https://mariusschulz.com/blog/function-definitions-in-javascript}


\subsubsection*{What JS engines does Chrome, Safari, Firefox, Edge use respectively?}

\subsubsection*{Difference between \code{const}, \code{let} and \code{var}}

\pagebreak

\addcontentsline{toc}{subsection}{CSS}
\subsection*{CSS}
\subsubsection*{CSS, SASS, LESS, CSS-in-JS}

\subsubsection*{What is Critical CSS}

\pagebreak

\addcontentsline{toc}{subsection}{Docker}
\subsection*{Docker}
\subsubsection*{What's the difference between a docker container and a virtual machine}

\subsubsection*{What is a docker repository and what is its use case}

\pagebreak

\addcontentsline{toc}{subsection}{TypeScript}
\subsection*{TypeScript}
\subsubsection*{What's the difference between using JavaScript and TypeScript in a project}

\subsubsection*{Difference between \code{type} and \code{interface}}

\subsubsection*{What are type assertions and when should you use them}

\subsubsection*{What are some disadvantages of TypeScript}

\subsubsection*{What are type generics used in TypeScript for}

\subsubsection*{What is the difference between using \code{null} and {unknown}}








\end{document}
